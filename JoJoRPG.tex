\documentclass[a4paper,12pt]{article}


\usepackage[english]{babel}
\usepackage{blindtext}
\usepackage{microtype}
\usepackage{graphicx}
\usepackage{wrapfig}
\usepackage{enumitem}
\usepackage{amsmath}
\usepackage{index}
\usepackage[table]{xcolor}
\usepackage{multirow}
\usepackage{array}
\usepackage{changepage}
\usepackage{dirtytalk}
\usepackage{multicol}
\usepackage{adjustbox}
\usepackage{hyperref}


\makeindex

\begin{document}
\title{\Large{\textbf{How to play JoJoRPG}}}
\author{By Gooon}

\maketitle


\section{General Information}
This informative texts assumes that you have some small amount of experience any kind of role-playing game, and that you at least have a superficial knowledge of the popular anime and manga series JoJo's Bizzare Adventure. For comprehensions sake, this document is segmented into several parts, each explaining one of the games general aspects, these being character-creation, out of combat gameplay, combat gameplay, and DMing.
\section{Character Creation}
To create a character to use in a session, or number of sessions of JoJoRPG follow the steps below. It is also of note that there are not yet systems for the creation or use of any characters harnessing the powers of Hamon or the Ripple implemented.
\begin{enumerate}
	\item Come up with a character. This step is in fact not just some recursive-meta-fuckery, but instead asking of you to simply generally outline your characters... character. What are they like? Where do they come from? What do they want? What is your characters appearance? And most importantly: What is your character Stand? What does it look like? What kind of powers does it posses?
	\item After properly outlining, and shaping a general idea of you player character, get in contact with your DM and tell them about your idea. This ensures that your DM a) gets a general feel for your character, and can, with your consent, change some parts of them, to more organically fit them into their concept of the sessions plot and its world, and b) is able to actually turn your abstract idea for the character into a playable set of game-mechanics.
	\item While conversing with your DM it is important to solidly and completely define the following traits:
	\begin{itemize}
		\item The characters, and their Stands name.
		\item A rough description, or even depiction, if possible, of your characters and their Stands appearance.
		\item A number of personality traits, preferably ones you yourself are comfortable role-playing.
		\item Your characters background. What are they normally doing? Do they have a job, or a family? Any relevant past experiences?
		\item Your characters stats. These are split into two sets, your characters overall stats, and their Stands stats. Your characters stats are Stamina, which is used up when your character takes damage or uses one of their moves or abilities, and Proficiency which represents your characters cumulative experience and how generally skilled they are at what the are doing. To determine these roll 2D20. You can then choose which roll you want to allocate to which stat. Then add 15 to your Stamina stat. Your Stands stats represent your stands capabilities in the aspects of Destructive Power, Speed, Range, Persistence, Precision and Development potential. What these stats individually describe is fairly self-evident. These stats come into play when their application is needed in an taken action, and modify the players roll. To determine your Stands stats, roll 5D6. You can then allocate these individual roles freely to all stats except Development Potential. That stat is determined by the DM while creating the scenario, to ensure that the players are on relatively equal footing when it comes to the session-overarching development of their character. Stand stats can change over time athough only very rarely, and only for a good reason.
	\end{itemize}
		\item After you and the DM have successfully created an outline for your character, it is now time to create your characters strengths and weaknesses. Ideally these are informed by your characters background, and harmonize with their personality. Note also that this step necessitates close collaboration with your DM to ensure that your character is neither comically overpowered, nor a helpless wreck. Your characters strengths and weaknesses are integral to playing them, as they come into play any time your character performs an action \footnote{more on that later}, modifying the result, either in your favour, or against you, with the severity of this modifier being decided case-to-case by the DM. All of these strengths and weaknesses will be written onto your character-sheet, for easy memorization.
		\item A similar process will then be repeated for your characters stand, albeit here you will properly define its ability or abilities. After properly defining your stands ability or abilities, it is now your task to come up practical applications, called moves in the context of the game. These moves will be the way through which you will predominantly use your abilities in game, think of them as your characters special techniques. Please note that this is a very story- and character-heavy game, and trying to come up with something explicitly broken abilities and moves to trivialize the games challenges is more often than not neither rewarding nor interesting, and will often cheapen the experience for all involves. It is advised to put limitations like \say{can only be used once per battle} or \say{can only be used once per game} on your moves in order to increase the abilities value. Your character can of course also use their abilities outside of the specific parameters and actions set by their moves, although it is advised to cover as many general use-cases of your ability or abilities with your moves, and reserve these applications for trivial or passive things, as not to clog up your character sheet or limit what you can do.. If this distinction between moves and abilities confuses you, take a look at the example character listed below.
		\item If you want to be especially faithful to the canon JoJo media, and choose to reference real-world music or other things in your character, and struggle to properly flesh them out, it is advised to take more than just the name of whatever you are referencing, but to let yourself be influenced by the works themes, aesthetic or concrete imagery.
		
\end{enumerate}


\end{document}