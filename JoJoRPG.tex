\documentclass[a4paper,12pt]{article}


\usepackage[english]{babel}
\usepackage{blindtext}
\usepackage{microtype}
\usepackage{graphicx}
\usepackage{wrapfig}
\usepackage{enumitem}
\usepackage{amsmath}
\usepackage{index}
\usepackage[table]{xcolor}
\usepackage{multirow}
\usepackage{array}
\usepackage{changepage}
\usepackage{dirtytalk}
\usepackage{multicol}
\usepackage{adjustbox}
\usepackage{hyperref}


\makeindex

\begin{document}
\title{\Large{\textbf{How to play JoJoRPG}}}
\author{By Gooon}

\maketitle


\section{General Information}
This informative texts assumes that you have some small amount of experience any kind of role-playing game, and that you at least have a superficial knowledge of the popular anime and manga series JoJo's Bizzare Adventure. For comprehensions sake, this document is segmented into several parts, each explaining one of the games general aspects, these being character-creation, out of combat gameplay, combat gameplay, and DMing.
\section{Character Creation}
To create a character to use in a session, or number of sessions of JoJoRPG follow the steps below. It is also of note that there are not yet systems for the creation or use of any characters harnessing the powers of Hamon or the Ripple implemented.
\begin{enumerate}
	\item Come up with a character. This step is in fact not just some recursive-meta-fuckery, but instead asking of you to simply generally outline your characters... character. What are they like? Where do they come from? What do they want? What is your characters appearance? And most importantly: What is your character Stand? What does it look like? What kind of powers does it posses?
	\item After properly outlining, and shaping a general idea of you player character, get in contact with your DM and tell them about your idea. This ensures that your DM a) gets a general feel for your character, and can, with your consent, change some parts of them, to more organically fit them into their concept of the sessions plot, and b) is able to actually turn your abstract idea for the character into a playable set of game-mechanics.
	\item While conversing with your DM it is important to properly define and gamify your character.
\end{enumerate}


\end{document}