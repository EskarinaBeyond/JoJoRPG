\documentclass[a4paper,12pt]{article}


\usepackage[english]{babel}
\usepackage{blindtext}
\usepackage{microtype}
\usepackage{graphicx}
\usepackage{wrapfig}
\usepackage{enumitem}
\usepackage{amsmath}
\usepackage{index}
\usepackage[table]{xcolor}
\usepackage{multirow}
\usepackage{array}
\usepackage{changepage}
\usepackage{dirtytalk}
\usepackage{multicol}
\usepackage{adjustbox}
\usepackage{hyperref}
\usepackage{longtable}
\usepackage[T1]{fontenc}
\usepackage[utf8]{inputenc}
\usepackage{array}



\makeindex

\begin{document}
\title{\Large{\textbf{How to play JoJoRPG}}}
\author{By Gooon}

\maketitle


\section{General Information}
This informative texts assumes that you have some small amount of experience any kind of role-playing game, and that you at least have a superficial knowledge of the popular anime and manga series JoJo's Bizzare Adventure. For comprehensions sake, this document is segmented into several parts, each explaining one of the games general aspects, these being character-creation, out of combat gameplay, combat gameplay, and DMing.
\section{Character Creation}
To create a character to use in a session, or number of sessions of JoJoRPG follow the steps below. It is also of note that there are not yet systems for the creation or use of any characters harnessing the powers of Hamon or the Ripple implemented.
\begin{enumerate}
	\item Come up with a character. This step is in fact not just some recursive-meta-fuckery, but instead asking of you to simply generally outline your characters... character. What are they like? Where do they come from? What do they want? What is your characters appearance? And most importantly: What is your character Stand? What does it look like? What kind of powers does it posses?
	\item After properly outlining, and shaping a general idea of you player character, get in contact with your DM and tell them about your idea. This ensures that your DM a) gets a general feel for your character, and can, with your consent, change some parts of them, to more organically fit them into their concept of the sessions plot and its world, and b) is able to actually turn your abstract idea for the character into a playable set of game-mechanics.
	\item While conversing with your DM it is important to solidly and completely define the following traits:
	\begin{itemize}
		\item The characters, and their Stands name.
		\item A rough description, or even depiction, if possible, of your characters and their Stands appearance.
		\item A number of personality traits, preferably ones you yourself are comfortable role-playing.
		\item Your characters background. What are they normally doing? Do they have a job, or a family? Any relevant past experiences?
		\item Your characters stats. These are split into two sets, your characters overall stats, and their Stands stats. Your characters stats are Stamina, which is used up when your character takes damage or uses one of their moves or abilities, Proficiency, which represents your characters cumulative experience and is used to upgrade and unlock abilities and Moves, and Movement, which determines how far your charqacter can move in a single action. To determine these roll 3D20. You can then choose which roll you want to allocate to which stat. Then add 15 to your Stamina stat and halve the Movement stat, rounding up. If your Movement stat is below 4, increase it to 4. Your Stands stats represent your Stands capabilities in the aspects of Destructive Power, Speed, Range, Persistence, Precision and Development potential. What these stats individually describe is fairly self-evident. These stats come into play when their application is needed in an taken action, and modify the players roll. To determine your Stands stats, roll 5D6. You can then allocate these individual roles freely to all stats except Development Potential, according to the following schema: 1 = E, 2 = D, 3 = C, 4 = C, 5 = B, 5 = A. Development Potential is determined by the DM while creating the scenario, to ensure that the players are on relatively equal footing when it comes to the session-overarching development of their character. Stand stats can change over time although only very rarely, and only for a good reason.
	\end{itemize}
		\item After you and the DM have successfully created an outline for your character, it is now time to create your characters strengths and weaknesses. Ideally these are informed by your characters background, and harmonize with their personality. Note also that this step necessitates close collaboration with your DM to ensure that your character is neither comically overpowered, nor a helpless wreck. Your characters strengths and weaknesses are integral to playing them, as they come into play any time your character performs an action \footnote{more on that later}, modifying the result, either in your favour, or against you, with the severity of this modifier being decided case-to-case by the DM. All of these strengths and weaknesses will be written onto your character-sheet, for easy memorization.
		\item A similar process will then be repeated for your characters stand, albeit here you will properly define its ability or abilities. After properly defining your stands ability or abilities, it is now your task to come up practical applications, called moves in the context of the game. These moves will be the way through which you will predominantly use your abilities in game, think of them as your characters special techniques. Please note that this is a very story- and character-heavy game, and trying to come up with something explicitly broken abilities and moves to trivialize the games challenges is more often than not neither rewarding nor interesting, and will often cheapen the experience for all involves. It is advised to put limitations like \say{can only be used once per battle} or \say{can only be used once per game} on your moves in order to increase the abilities value. Your character can of course also use their abilities outside of the specific parameters and actions set by their moves, although it is advised to cover as many general use-cases of your ability or abilities with your moves, and reserve these applications for trivial or passive things, as not to clog up your character sheet or limit what you can do.. If this distinction between moves and abilities confuses you, take a look at the example character listed below.
		\item If you want to be especially faithful to the canon JoJo media, and choose to reference real-world music or other things in your character, and struggle to properly flesh them out, it is advised to take more than just the name of whatever you are referencing, but to let yourself be influenced by the works themes, aesthetic or concrete imagery.
		
\end{enumerate}
It is also customary for the  DM to create a list of future upgrades for all characters together with their respective players. These upgrades are unlocked once the character in question crosses a specified threshold of Proficiency. It is further advised for the DM and their players to retool and adapt their character between sessions, if players have grown attached to their characters, and whish to carry them over into the next adventure. This can kink out any awkwardness in the characters play style, and is able to account for time-skips, or other off-screen character development, although only to a defree the DM agrees to.\\
The following is an example character to further demonstrate the act character creation. The files for an empty character sheet of this design are also in this repository.

\pagebreak

		\begin{longtable}{| p{6cm} | p{6cm} |}
		\hline
		\multicolumn{2}{| c |}{ \textbf{\LARGE Frederick "Freddy" Queensly}} \\
		\hline
		\large Stamina: [ ]/25 & \large Proficiency: 10  \\
		\hline
		\multicolumn{2}{| p{12cm} |}{\large Movement: 5 } \\
		\hline
		\multicolumn{2}{| p{12cm}| }{Description: A middle-aged middle-manager. Shamelessly leverages his position of power in the workplace against his subordinates, and enjoys the suffering he can cause them to experience using his stand. Talks a big game, but failed all the way up. Rude an short-tempered. Wants to start a family to \say{prove his manhood}, and aggressively approaches any women he can to achieve this goal. Traditional values(read: deeply bigoted). \say{Call me Freddy (: }}\\
		\hline
		\large Strengths: & \large Weaknesses: \\
		\hline
		\begin{itemize}
			\item People Pleaser: +2 on rolls to convince or persuade a person.
			\item Lucky Bastard: Whenever you roll  a 1, roll again. The critical failure doesn not comes into effect on roll of 15 or higher
		\end{itemize} 
		&
		\begin{itemize}
		\item Traditional Values: Freddy aggressively underestimates anyone he sees as \say{inadequate}, e.g anyone who is not white, male and straight. -5 on defence roles against any character that is not that.
		\item Office Body: Whenever you use more than one movement-action in one turn, it costs 3 Stamina. 
		\end{itemize}\\
		\hline
		\multicolumn{2}{| c |}{\LARGE \textbf{ Under Pressure}} \\
		\hline
		\multicolumn{1}{| c }{\large Destructive Power: B} & \multicolumn{1}{| c |}{\large Speed: C} \\
		\hline
		\multicolumn{1}{| c }{\large Range: B}  &\multicolumn{1}{| c |}{\large Persistence: B} \\
		\hline
		\multicolumn{1}{| c }{\large Precision: D}  & \multicolumn{1}{| c |}{\large Development Potential: E} \\
		\hline
		\multicolumn{2}{| p{12cm}| }{Description: Takes the appearance of a fanged skeleton clad in a red cape that looks as if it is frozen mid-explosion. }\\
		\hline 
		\multicolumn{2}{| p{12cm} | }{ Abilities: Under Pressure can perpetuate and increase stress, fear and anxiety in any target within its range of 50m. It can then convert these stored up negative emotions into physical explosions. These explosions do not harm the target they hail from, but inflict high damage in the targets surroundings, with the size and intensity of these explosions being relative to the targets negative emotions. Freddy can also use the ability on himself, and also has an amount of control over these explosions. He can choose to have the target, \say{vent} their negative emotions through the explosions.}\\	
		\hline
		\multicolumn{2}{| c |}{\large Moves} \\
		\hline
		\multicolumn{2}{| p{12cm} |}{
			\begin{itemize}
				\item (3) - The Terror Of Knowing - The targets negative emotions are increased dramatically. If the target had no negative emotions before the attck, add 1 Negative Emotions Marker to the target, otherwise, double the amount of Negative Emotions Markers. If the successfully target is a player character, they must also act accordingly. On their turn a hit target can use one or more of their actions to calm themselves and remove some of their Negative Emotions Markers, how many is decided by the roll of a d20 and the DM. If X is the amount of Negative Emotions Markers on the target, they have a -X/2 modifier on all rolls, rounding down.
				\item (5) - What The World Is About - The targets negative emotions manifest themselves in violent explosion. If X is the amount of Negative Emotions Markers on the target, everything in a radius of x/2 m takes X damage.
				\item (3) - Keep Coming Up With Love - You can relief some of your negative emotions by venting them through your explosions. You have a +2 modifier on your next roll. Everything in a 2, radius around you takes 5 damage.
				\item (3) - But Its So Slashed And Torn - Whenever you take damage you can use this move to create a small explosion, to let out the anger you feel over getting hit. Hits one target, up to 4m away, 3 damage.
			\end{itemize}}\\
		\hline
	
		\end{longtable}



\section{Out of combat Gameplay} 
Any scenario that is not explicitly stated to be combat counts as out of combat. This means that the players are free to converse with any characters, and interact with any thing within the current location. Any action that can not be described as being trivial however, still requires a roll to succeed, with the rolls difficulty or any other parameters being made up by the DM as they go along. Every Move or Ability you can use in combat, you can also use out of combat, gives that its use makes sense and that the DM allows it , which they most likely will.

\section{Combat Rules}
Any scenario that is started with the magical words \say{Roll for initiative} counts a combat Rolling for initiative means that all participants roll a d20 and add their movement value. A turn-order is then formed beginning with the character with the highest outcome, and ending with the character with the lowest. Ties are sorted out using individual roll-offs.\\
On their turn any character can take a number of actions. The standard number is 2, although this can change in accordance with your DMs preferences, and/or the level of escalation in the story. Many actions requires the rolling of a dice to determine their success. This is most commonly done with a standard d20 dice. Whenever a player is asked to roll for an action, the roll has a set value, the difficulty, which must be overcome. Rolls can be argumented with various modifiers, like the your characters stats, or any other bonuses or maluses accumulated during combat. The general rule of thumb for \say{normal} actions is as follows:
\begin{itemize}
	\item nat 1: Rolling a 1 without any modifiers results in a critical failure, in which the DM dictates in what specific way the action goes wrong, and the consequences of its failure.
	\item 1-4: if the final result is between 1 and 5, the actions simply fails, with only few bad consequences.
	\item 5-9: the action succeeds, but not fully. The bear minimum happens, albeit with some negative consequences.
	\item 10-16: You succeed, you did what you set out to do, although little more.
	\item 17+: Great success, you not only accomplish your task, but are also rewarded with some additional positive consequences.
	\item nat 20: Rolling a 20 without any modifier results in a critical success, in which you dictate in what specific way your actions benefits you and your party.
\end{itemize}
Please note that the listing above is nothing but a vague guideline, and that the ultimate consensus about what succeeds and what does not lies by the DM, and that the distribution of possible outcomes can be skewed in any direction by the actions difficulty. DMs are advised to benefit players that seem to be out of luck for the day, by letting them succeed, with more ease, after several consecutive low rolls.
The types of actions a character can take are listed below:
\begin{itemize}
	\item Moving: The character an amount of meters equal or below their Movement stat. Climbing, jumping or other special maneuver does not consume an additional action, but requires a roll to determine its success.
	\item Attacking and Defending: Whenever a character tries to attack, their player say how they want to attack, e.g. with what weapon, if any, or where specifically on the target. This also applies to Stands. Then the attacker rolls a d20 and adds any relevant modifiers to the result. If the attacks difficulty, as determined has been bested, the attack succeeds. The attacked character now has a chance to defend against the attack in the same manner. Please note that attacks are argumented by abilities, strengths or weaknesses, not just in the form of simple modifiers, but also in more esoteric ways, such as their personal style of doing things. Bottom line: stay in character!
	\item Using a Move: Same as attacking, but performed within the mechanical constraints of the chosen Move. Using a Move takes up the specified amount of Stamina.
	\item Using an Ability: Simply using an defined Stand-Ability to perform an action. Difficulty and amount of consumed Stamina are determined by the DM.
	\item Readying an action. You can also choose to ready an action, and set it to a specific trigger(e.g when another character moves into your range, or once some occurrence has subsided). Your turn will then be put on hold, until the specified trigger is activated, and you perform your action. A readied action can be interrupted by an attack, or any other occurrence the DM deems appropriate.
	\item Combining Actions: Two or more characters can combine their actions into one coordinated action. To do this the characters have to coordinate themselves in-game. The action takes place on the turn of the participant with the lowest rank of the turn order, relative to the initiator of the combined action. Every participant has to roll individually for their contribution to the combined action, with each of their successes and failures accumulating into the overall success or failure of the action.
\end{itemize}
If a player characters Stamina is reduced to to 0 or lower during combat they are unconscious. If the
After successful combat it is advised to reward the players with stat-ups, and other goodies, and to play some sweet victory music. Additionally, characters are healed by an amount the DM deems fit, and any unconscious characters wake up. 




\end{document}